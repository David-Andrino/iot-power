\section{Introducción}

\subsection{Descripción funcional del equipo}

El sistema se encarga de cargar mediante energía solar y monitorizar dos baterías $12 V$ / $7 Ah$. Se encargará de guardar las medidas en un log y se visualizarán en un servidor, al que el usuario podrá acceder mediante un PC o un móvil. Además, el sistema dispondrá de una batería de backup para alimentar el sistema cuando la situación lo requiera (Bajo rendimiento solar). La batería de backup también se recargará mediante energía solar.


\subsection{Especificaciones funcionales}

Se diseñará un sistema de carga de baterías para emplazamientos aislados exteriores, que utilice energía solar como fuente de energía y que sea controlable y monitorizable a través de internet. El sistema deberá cumplir las siguientes especificaciones técnicas para asegurar su correcto y completo funcionamiento:

\subsubsection{Especificaciones Hardware}

\begin{itemize}
    \item El sistema deberá cargar dos baterías a la vez (\texttt{BAT1} y \texttt{BAT2}).
    \item Se utilizará un circuito específico para la carga de las baterías que debe contar con protección.
    \item El sistema priorizará el uso de la energía generada por el panel solar.
    \item El sistema contará con una tercera batería de backup, que se utilizará cuando el panel solar sea incapaz de entregar la energía necesaria.
    \item La batería de backup se cargará también mediante energía solar.
    \item Se utilizarán tres baterías de plomo ácido de $12 V$ y $7 Ah$ cada una. 
    \item El panel solar debe ser capaz de generar suficiente energía para cargar ambas baterías en condiciones ideales.
    \item Se utilizará el microcontrolador \texttt{ESP8266} para el control y monitoreo del sistema mediante \texttt{I2C} enviar las medidas por \texttt{Wi-Fi}.
    \item Se utilizará este sensor \texttt{INA226} para medir la tensión y corriente de cada batería y el panel solar.
    \item El sistema debe ser resistente a condiciones climáticas adversas. 
    \item El sistema deberá contar con un diseño que permita la adecuada ventilación y disipación del calor generado por la electrónica.
    \item Se minimizará al máximo el consumo del sistema encargado del control de las baterías.
    \item El sistema cumplirá las normativas de seguridad eléctrica y medioambiental.
    \item Se realizará pruebas de integración de todo el sistema.
    \item El sistema se conectará a una red \texttt{WiFi} con acceso a un servidor.
    \item El sistema enviará las medidas a un servidor externo.
\end{itemize}

\subsubsection{Especificaciones Software}

\begin{itemize}
    \item Se utilizará el entorno \texttt{Arduino IDE} para el desarrollo del código en el lenguaje \texttt{Arduino}.
    \item Se configurará y gestionará la conexión \texttt{WiFi} en el \texttt{ESP8266}.
    \item Toda la información estará disponible al usuario mediante un servidor web y un log almacenado en la memoria \texttt{flash} del microcontrolador.
    \item En caso de pérdida de señal, el sistema seguirá almacenando medidas localmente.
    \item Se dispondrá de un modo de lectura de medidas para obtener el log de medidas a través de \texttt{Serial}.
    \item Se utilizará el protocolo de comunicación \texttt{MQTT} para la comunicación entre el \texttt{ESP8266} y el servidor.
    \item El cliente podrá acceder al sistema a través de una interfaz web (en móvil u ordenador).
    \item En el servidor web se visualizará en tiempo real gráficas y valores que correspondan con el estado de carga de las baterías y la producción de energía del panel solar.
    \item En el log se almacenará la tensión y corriente de carga de las tres baterías y la producción de energía del panel solar.
    \item El funcionamiento normal del sistema consistirá en: El sistema se despierta, intenta conectarse a la \texttt{WiFi}, toma medidas, las gestiona (envío y almacenamiento) y se duerme. De esta forma se optimiza el consumo de batería.
    \item Se desarrollarán pruebas para cada módulo software.
\end{itemize}
