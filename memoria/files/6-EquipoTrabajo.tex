\section{Equipo de trabajo}

Todos los integrantes hemos colaborado en el diseño de los módulos, pero la responsabilidad principal de cada uno se ha repartido como:

\begin{itemize}
	\item Rubén Agustín Gonzalez: Interfaz de la pantalla de la placa, bajo consumo y uSD.
	\item David Andrino Izquierdo: Esquemático de la alimentación, diseño de las PCBs, módulo de control, protector I2C y procesado digital de señal.
	\item Estela Mora Barba: Esquemático del amplificador de audio, módulo NFC y uSD.
	\item Fernando Sanz Giménez: Módulo RTC, radio, MP3 y web.
\end{itemize}

Las siguientes partes han sido relacionadas de forma conjunta por todos:
\begin{itemize}
	\item Integración del proyecto.
	\item Realización de la memoria.
	\item Powerpoints de las presentaciones.
\end{itemize}

A la hora de colaborar y juntar todas las partes del proyecto, hemos utilizado un repositorio en GitHub \footnote{\url{https://github.com/David-Andrino/ise-rtap}}, con varias ramas creadas para cada miembro. Además, hemos estado trabajando tanto presencial como online, de forma tanto individual como colectiva. Ha habido mucha colaboración entre los miembros del grupo, no solo en lo referente al propio trabajo sino que también de forma emocional. 