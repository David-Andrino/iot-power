\documentclass{article}
\begin{document}
\subsection{Modulo Principal}

Este módulo está formado por el archivo, \texttt{MAIN\_POWER.ino} que se encarga de conectar todos los módulos en un sistema unificado.

El fichero contiene las siguientes funciones:

- \texttt{setup()}: Inicializa los pines que usamos para la comunicación I2C y los relés, la velocidad del puerto serie y el valor inicial de la estructura de datos que contienen.

- \texttt{loop()}: Realiza todas las operaciones de funcionamiento del sistema. La secuencia de ejecución es la siguiente:
\begin{enumerate}
    \item Recoge las medidas de losz INA.
    \item Revisa los valores obtenidos del panel solar por si es necesario cambiar la alimentación del panel solar a la de backup.
    \item Se intenta conectar a la red Wifi. Si lo consigue se intenta mandar las medidas al servidor mqtt.
    \item Guarda las medidas de los sensores en un fichero.
    \item Duerme al ESP poniéndolo en bajo consumo.
    \item Sale del modo de bajo consumo.
\end{enumerate}

\end{document}
