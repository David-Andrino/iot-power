La librería utilizada para implementar el cliente MQTT ha sido la librería \texttt{PubSubClient}.  Esta contiene toda la información necesaria para la creación de un cliente \texttt{MQTT}, la conexión con un servidor \texttt{MQTT} y la publicación y recepción de mensajes de servidores \texttt{MQTT}.

En nuestro proyecto hemos utilizado esta herramienta únicamente para la conexión con el servidor \texttt{MQTT} y el envío de los datos de las medidas de los \texttt{INA}.

La versión utilizada es la \texttt{2.8} y pueden obtenerse los ficheros fuente en su repositorio en \texttt{GitHub}. \cite{olearyKnollearyPubsubclient2024}

Adicionalmente, se cuenta con una página con documentación completa de esta librería y su funcionamiento. \cite{nickolearyArduinoClientMQTT}