Una de las librerías utilizadas para realizar el proyecto ha sido la librería \texttt{INA226\_WE}.  Esta contiene toda la información necesaria para la comunicación \texttt{I2C}, configuración y obtención de datos de los sensores \texttt{INA226}, además de otras herramientas útiles como el establecimiento del modo de bajo consumo para los sensores.

En nuestro proyecto hemos utilizado esta herramienta para obtener los datos de tensión y corriente de los 4 \texttt{INA}.  Aunque adicionalmente, se han utilizado las herramientas de configuración para ajustar de manera correcta los sensores y que estas medidas obtenidas sean fieles a la realidad, además de utilizar su modo de bajo consumo cuando no era necesario obtener medidas.

La versión utilizada es la \texttt{1.2.9} y pueden obtenerse los ficheros fuente en su repositorio en \texttt{GitHub}. \cite{ewaldWollewaldINA226_WE2024}

Adicionalmente, se cuenta con una página con documentación completa de esta librería y su funcionamiento. \cite{ewaldINA226CurrentPower2021}