\documentclass{article}
\begin{document}
\subsubsection{Modulo Cliente MQTT}

El módulo MQTT es el encargado de crear un cliente MQTT y establecer una conexión con el servidor MQTT para enviar los datos de la ultima medida que se haya realizado. 

Este módulo está formado por dos archivos:
\begin{itemize}
    \item \texttt{mqtt\_client.cpp} : Contiene las funciones necesarias para la conexión.
    \item \texttt{mqtt\_client.hpp} : Contiene la definición de las funciones, la importación de las librerías externas utilizadas y el número de reintentos permitidos al realizar una conexión con el servidor.
\end{itemize}

Para la implementación de este módulo se ha requerido la utilización de las librerías \texttt{“PubSubClient.h”}, que permite la conexión MQTT, \texttt{“ESP8266WiFi.h”} y \texttt{“WiFiUdp.h”} para configurar el cliente UDP.
%\TODO{ https://github.com/knolleary/pubsubclient }

El fichero \texttt{mqtt\_client.cpp} contiene las siguientes funciones:

-\texttt{reconnect(PubSubClient\& client, char* const mqtt\_server, int mqtt\_port)}: Para intentar una reconexión con el servidor MQTT en caso de que haya fallado. Aunque este intento de reconexión es limitando.

-\texttt{bool publishTelemetry(char* const mqtt\_server, int mqtt\_port, telemetry\_t\& telemetry)}: Intenta publicar un mensaje con todos los datos medidos por el INA al servidor MQTT.  La secuencia de ejecución es la siguiente:
\begin{enumerate}
    \item Crea un cliente MQTT.
    \item Crea una conexión con el servidor MQTT.
    \item Verifica si el cliente se ha podido conectar con el servidor. Si no ha podido intenta realizar una reconexión en el caso de que haya habido un error.
    \item Formatea el mensaje para que pueda ser procesado por el servidor.
    \item Publica el mensaje en el servidor y este devuelve un booleano que indica si el envío ha sido correcto.
    \item Se desconecta el cliente.
    \item Devuelve el booleano.
\end{enumerate}

\end{document}
