\subsubsection{Modulo Cliente \texttt{MQTT}}

El módulo \texttt{MQTT} es el encargado de crear un cliente \texttt{MQTT} y establecer una conexión con el servidor \texttt{MQTT} para enviar los datos de la última medida que se haya realizado. 

Este módulo está formado por dos archivos:
\begin{itemize}
    \item \texttt{mqtt\_client.cpp}: Contiene las funciones necesarias para la conexión.
    \item \texttt{mqtt\_client.hpp}: Contiene la definición de las funciones, la importación de las librerías externas utilizadas y el número de reintentos permitidos al realizar una conexión con el servidor.
\end{itemize}

Para la implementación de este módulo se ha requerido la utilización de las librerías \texttt{PubSubClient.h}, que permite la conexión \texttt{MQTT}, \texttt{ESP8266WiFi.h} y \texttt{WiFiUdp.h} para configurar el cliente UDP. \cite{olearyKnollearyPubsubclient2024}

El fichero \texttt{mqtt\_client.cpp} contiene las siguientes funciones:

\begin{itemize}

\item \texttt{reconnect(PubSubClient\& client, char* const mqtt\_server, int mqtt\_port)}

Intentar una reconexión con el servidor \texttt{MQTT} en caso de que haya fallado. Se realiza un número limitado de reintentos antes de desistir. 

\begin{lstlisting}[language=c++,caption={Función \texttt{reconnect}},captionpos=b]
void reconnect(PubSubClient& client, char* const mqtt_server, int mqtt_port) {
    for (int i = 0; i < 3 && !client.connected(); i++){

        Serial.print("[MQTT] Iniciando conexion a " + String(mqtt_server) + ":" + String(mqtt_port));

        if (client.connect("Cliente", "Cliente", "Cliente")) {
            Serial.println("[MQTT] Conectado");
        } else {
            Serial.print("[MQTT] Conexion fallida. Estado = " + String(client.state()) + ". Reintentando en 0.5 segundos...");
            delay(500);
        }
    }
}
\end{lstlisting}

\item \texttt{publishTelemetry(char* const mqtt\_server, int mqtt\_port, telemetry\_t\& telemetry)}. 

Intenta publicar un mensaje con todos los datos medidos por el \texttt{INA} al servidor \texttt{MQTT}.  La secuencia de ejecución es la siguiente:
\begin{enumerate}
    \item Crea un cliente \texttt{MQTT}.
    \item Crea una conexión con el servidor \texttt{MQTT}.
    \item Verifica si el cliente se ha podido conectar con el servidor. Si no ha podido intenta realizar una reconexión en el caso de que haya habido un error.
    \item Formatea el mensaje para que pueda ser procesado por el servidor.
    \item Publica el mensaje en el servidor y este devuelve un booleano que indica si el envío ha sido correcto.
    \item Se desconecta el cliente.
    \item Devuelve el booleano.
\end{enumerate}

\begin{lstlisting}[language=c++,caption={Funcion \texttt{publishTelemetry}},captionpos=b]
bool publishTelemetry(char* const mqtt_server, int mqtt_port, telemetry_t& telemetry){
    WiFiClient wifiClient;
    WiFiUDP ntpUDP;
    PubSubClient client(wifiClient);
    client.setServer(mqtt_server, mqtt_port);
    bool statusPublish=false;
    char result[256];

    if (!client.connected()){
        reconnect(client, mqtt_server, mqtt_port);
    }

    snprintf(result, sizeof(result), "{VSolar:%.2f,ISolar:%.2f,VBatbu:%.2f,IBatbu:%.2f,"
        "VBat1:%.2f,IBat1:%.2f,VBat2:%.2f,IBat2:%.2f}", 
        telemetry.VSolar, telemetry.ISolar, telemetry.VBatbu, telemetry.IBatbu, 
        telemetry.VBat1, telemetry.IBat1, telemetry.VBat2, telemetry.IBat2);

    statusPublish = client.publish("v1/devices/me/telemetry" ,result);
    client.disconnect();
    return statusPublish;
}
\end{lstlisting}

\end{itemize}
