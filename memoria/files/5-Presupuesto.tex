\section{Presupuesto}
En la \autoref{tab:presupuesto} se pueden ver los gastos del proyecto.

\begin{table}[H]
    \centering
    \begin{tabular}{lrcr}
    \toprule
    \multicolumn{1}{c}{\textbf{Componente}}  & {\textbf{Precio unitario}}   & {\textbf{Cantidad}} & {\textbf{Precio total}}   \\ \midrule
    \textbf{Batería}                         & {$12.00$ \euro}              & {2}                 & {$24.00$ \euro}           \\ 
    \textbf{CN3768}                          & {$5.89$ \euro}               & {3}                 & {$17.67$ \euro}           \\ 
    \textbf{LDO AMS1117}                     & {$2.05$ \euro}               & {1}                 & {$2.05$ \euro}            \\ 
    \textbf{INA226}                          & {$1.75$ \euro}               & {4}                 & {$7.00$ \euro}            \\ 
    \textbf{PS58F1}                          & {$4.29$ \euro}               & {1}                 & {$4.29$ \euro}            \\ 
    \textbf{LM2596}                          & {$0.61$ \euro}               & {1}                 & {$0.61$ \euro}            \\ 
    \textbf{XL6009}                          & {$0.79$ \euro}               & {1}                 & {$0.79$ \euro}            \\ 
    \textbf{Componentes para el montaje}     & {$20.00$ \euro}              & {N/A}               & {$20.00$ \euro}           \\ 
    \textbf{PCB relé}                        & {$1.00$ \euro}               & {5}                 & {$5.00$ \euro}            \\ 
    \textbf{Componentes relé}                & {$5.00$ \euro}               & {N/A}               & {$5.00$ \euro}            \\ \midrule
    \textbf{Total}                           & { }                          & { }                 & {$86.41$ \euro}           \\ \bottomrule
    \end{tabular}
    \caption{Presupuesto del proyecto}
    \label{tab:presupuesto}
\end{table}