\section{Código de la aplicación}

\begin{lstlisting}[language=c++,caption={Fichero \texttt{file\_management.hpp}},captionpos=b]
/**
* @file file_management.hpp
* @author Estela Mora
* @author David Andrino
* @author Hugo Sanchez
* @author Fernando Sanz
* @brief Modulo de gestion de ficheros
* @version 1.1
* @date 15-12-2024
* 
* @copyright Copyright (c) 2024
* 
*/
#ifndef FILE_MANAGEMENT_H
#define FILE_MANAGEMENT_H

#include <FS.h>
#include <time.h>
#include "telemetry.hpp"

/**
 * @brief Borra el contenido del fichero indicado.
 * 
 * @param measFile Nombre del fichero a borrar.
 */
extern void clear_file(const char* measFile);

/**
 * @brief Lee el contenido del fichero indicado. Imprime en consola el contenido.
 * 
 * @param measFile Nombre del fichero a leer.
 */
extern void read_meas(const char* measFile);

/**
 * @brief Escribe la medida en el fichero indicado.
 * Formato: [timestamp] VPanelSolar: X.XX V, IPanelSolar: X.XX mA, VBatBackup: X.XX V, IBatBackup: X.XX mA, VBat1: X.XX V, IBat1: X.XX mA, VBat2: X.XX V, IBat2: X.XX mA
 * 
 * @param measFile Nombre del fichero.
 * @param measures Medidas a escribir.
 * @param timestamp Marca de tiempo.
 * @return true si la escritura fue exitosa, false en caso contrario.
 */
extern bool write_meas(const char* measFile , telemetry_t measures, String timestamp);

#endif
\end{lstlisting}

\begin{lstlisting}[language=c++,caption={Fichero \texttt{file\_management.cpp}},captionpos=b]
#include "file_management.hpp"

void clear_file(const char* measFile) {
    if (!SPIFFS.begin()) {
        Serial.println("[FILE_MGT] Error iniciando SPIFFS");
        return;
    }

    File file = SPIFFS.open(measFile, "w");
    if (!file) {
        Serial.println("[FILE_MGT] Error abriendo el fichero para borrar");
        return;
    }

    file.close();
    Serial.println("[FILE_MGT] El contenido del fichero ha sido borrado.");
}

void read_meas(const char* measFile){
    if (!SPIFFS.begin()) {
        Serial.println("[FILE_MGT] Error iniciando SPIFFS");
        return;
    }

    File measurementFile = SPIFFS.open(measFile, "r");
    if (!measurementFile) {
        Serial.println("[FILE_MGT] Error abriendo fichero");
        return;
    }

    Serial.println("[FILE_MGT] Abriendo " + String(measFile) + ". Contenido: ");

    while (measurementFile.available()) {
        Serial.write(measurementFile.read());
    }

    measurementFile.close();
}

bool write_meas(const char* measFile, telemetry_t measures, String timestamp) {
    if (!SPIFFS.begin()) {
        return false;
    }

    File measurementFile = SPIFFS.open(measFile, "a");
    if (!measurementFile) {
        return false;
    }

    String measurement = timestamp +
                         " VPanelSolar: " + String(measures.VSolar) + " V, " +
                         "IPanelSolar: " + String(measures.ISolar) + " mA, " +
                         "VBatBackup: " + String(measures.VBatbu) + " V, " +
                         "IBatBackup: " + String(measures.IBatbu) + " mA, " +
                         "VBat1: " + String(measures.VBat1) + " V, " +
                         "IBat1: " + String(measures.IBat1) + " mA, " +
                         "VBat2: " + String(measures.VBat2) + " V, " +
                         "IBat2: " + String(measures.IBat2) + " mA";

    measurementFile.println(measurement);
    measurementFile.close();

    return true;
}
\end{lstlisting}

\begin{lstlisting}[language=c++,caption={Fichero\texttt{hardware.hpp}}, captionpos=b]
/**
* @file hardware.hpp
* @author Estela Mora
* @author David Andrino
* @author Hugo Sanchez
* @author Fernando Sanz
* @brief Definicion de pines hardware
* @version 1.1
* @date 15-12-2024
*/
#ifndef HARDWARE_H
#define HARDWARE_H

#define RELAY_SW_IN 13
#define RELAY_SW_OUT 15
#define SDA 4
#define SCL 5

#endif
\end{lstlisting}

\begin{lstlisting}[language=c++,caption={Fichero\texttt{ina.cpp}}, captionpos=b]
//! Clase ina. Engargada de configurar y solicitar medidas al los INAs del circuito.

#include "ina.hpp"

// INA226 instances
INA226_WE Solar(I2C_D_PANEL);
INA226_WE Batbu(I2C_D_BAT_BU);
INA226_WE Bat1(I2C_D_BAT_1);
INA226_WE Bat2(I2C_D_BAT_2);

//! Booleano.
/*! Indica si se ha realizado la configuracion inicial de los sensores*/
bool first_config=false;

void setupINA226Sensors();
void reconfig_INAs();
void powerDownINA226();
void powerUpINA226();

/**************************************************************************//**
 * Configuracion inicial de los sensores.
 *
 * Tras un reset se inician los sensores para ser modificados, estableciendose los parametros:
 *  -Tiempo de conversion: Entre 140 us y 8.244 ms.
 *  -Modo de medida: POWER_DOWN (Sistema apagado), TRIGGERED (medidas a peticion) o CONTINUOUS (medidas constantes).
 *  -Valor de la resistencia se SHUNT. 
 *  -Rango de corrientes.
 *  -Factor de correcion.
 ******************************************************************************/
void setupINA226Sensors() {
    Wire.begin();
    Solar.init();
    Batbu.init();
    Bat1.init();
    Bat2.init();
}

/**************************************************************************//**
 * Reconfiguracion de los sensores.
 *
 * Al desperta a los sensores, se reinician con la configuracion por defecto.
 * Por ello es necesario reconfigurarlos al volver a tomar medidas.
 ******************************************************************************/

void reconfig_INAs(){
    Solar.setConversionTime(CONV_TIME_1100);
    Batbu.setConversionTime(CONV_TIME_1100);
    Bat1.setConversionTime(CONV_TIME_1100);
    Bat2.setConversionTime(CONV_TIME_1100);

    Solar.setMeasureMode(CONTINUOUS);
    Batbu.setMeasureMode(CONTINUOUS);
    Bat1.setMeasureMode(CONTINUOUS);
    Bat2.setMeasureMode(CONTINUOUS);
    
    Solar.setResistorRange(0.1, 10);
    Batbu.setResistorRange(0.1, 10);
    Bat1.setResistorRange(0.1, 10);
    Bat2.setResistorRange(0.1, 10);
    
    // Factores de correccion medidos experimentalmente
    Solar.setCorrectionFactor(1.0469);
    Batbu.setCorrectionFactor(1.0419);
    Bat1.setCorrectionFactor(0.9650);
    Bat2.setCorrectionFactor(0.9624);
    
    Solar.waitUntilConversionCompleted();
    Batbu.waitUntilConversionCompleted();
    Bat1.waitUntilConversionCompleted();
    Bat2.waitUntilConversionCompleted();
}

/**************************************************************************//**
 * Realiza una medida de todos los sensores.
 *
 *Inicialmente se verifica si se ha realizado la configuracion inicial de los INAs.
 *  -Si se ha realizado, se despiertan y reconfiguran
 *  -En caso contrario se realiza una configuracion inicial antes de la captacion de datos.
 *Se realiza 1 medida que es almacenada en la estrucutra de datos.
 *Y finalmente se ponen a bajo consumo los INA.
 * @param telemetry Puntero a la estructura de datos que contiene las medidas de tension y corriente de las baterias y panel solar.
 ******************************************************************************/

void measureINA226(telemetry_t *telemetry) {
    // Panel Solar
    if(first_config==false){
      setupINA226Sensors();
      first_config=true;
    }else{
      powerUpINA226();  
    }

    reconfig_INAs();
    
    Solar.readAndClearFlags();
    telemetry->VSolar = Solar.getBusVoltage_V() + (Solar.getShuntVoltage_mV() / 100);
    telemetry->ISolar = Solar.getCurrent_mA();

    // Bateria Backup
    Batbu.readAndClearFlags();
    telemetry->VBatbu = Batbu.getBusVoltage_V() + (Batbu.getShuntVoltage_mV() / 100);
    telemetry->IBatbu = -Batbu.getCurrent_mA();

    // Bateria 1
    Bat1.readAndClearFlags();
    telemetry->VBat1 = Bat1.getBusVoltage_V() + (Bat1.getShuntVoltage_mV() / 100);
    telemetry->IBat1 = -Bat1.getCurrent_mA(); // Invertir signo porque esta al reves

    // Bateria 2
    Bat2.readAndClearFlags();
    telemetry->VBat2 = Bat2.getBusVoltage_V() + (Bat2.getShuntVoltage_mV() / 100);
    telemetry->IBat2 = Bat2.getCurrent_mA();
    powerDownINA226();
}

/**************************************************************************//**
 * Pone en modo de bajo consumo a los INA.
 *
 * Tras una medida no es necesario que se mantengan activos los sensores por lo que es necesario ponerlos a bajo consumo.
 ******************************************************************************/

void powerDownINA226() {
    Solar.powerDown();
    Batbu.powerDown();
    Bat1.powerDown();
    Bat2.powerDown();
}

/**************************************************************************//**
 * Despierta a los INA del bajo consumo.
 *
 * Al desperta a los sensores, se reinician con la configuracion por defecto.
 * Por ello es necesario reconfigurarlos al volver a tomar medidas.
 ******************************************************************************/

void powerUpINA226() {
    Solar.powerUp();
    Batbu.powerUp();
    Bat1.powerUp();
    Bat2.powerUp();
}
\end{lstlisting}

\begin{lstlisting}[language=c++,caption={Fichero\texttt{ina.hpp}}, captionpos=b]
/**
* @file ina.hpp
* @author Estela Mora
* @author David Andrino
* @author Hugo Sanchez
* @author Fernando Sanz
* @brief Modulo de medida de sensores INA226
* @version 1.1
* @date 15-12-2024
*/
#ifndef ina_h
#define ina_h

#include <Wire.h>
#include <INA226_WE.h>
#include "telemetry.hpp"

/** @def I2C_D_PANEL
    @brief Indica la direccion I2C del INA que toma las medidas del panel solar.
*/
#define I2C_D_PANEL  0x45
/** @def I2C_D_BAT_BU
    @brief Indica la direccion I2C del INA que toma las medidas de la bateria de BackUp.
*/
#define I2C_D_BAT_BU 0x40
/** @def I2C_D_BAT_1
    @brief Indica la direccion I2C del INA que toma las medidas de la primera bateria.
*/
#define I2C_D_BAT_1  0x44
/** @def I2C_D_BAT_2
    @brief Indica la direccion I2C del INA que toma las medidas de la segunda bateria.
*/
#define I2C_D_BAT_2  0x41

/**
 * @brief Realiza una medida de todos los sensores.
 *
 * Inicialmente se verifica si se ha realizado la configuracion inicial de los INAs.
 *     - Si se ha realizado, se despiertan y reconfiguran.
 *     - En caso contrario se realiza una configuracion inicial antes de la captacion de datos.
 * Se realiza 1 medida que es almacenada en la estructura de datos.
 * Finalmente se apagan los INA.
 * 
 * @param telemetry Puntero a la estructura de datos que contiene las medidas.
 */
void measureINA226(telemetry_t *telemetry);

#endif
\end{lstlisting}

\begin{lstlisting}[language=c++,caption={Fichero\texttt{MAIN\_POWER.ino}}, captionpos=b]
#include "telemetry.hpp"
#include "hardware.hpp"
#include "ina.hpp"
#include "mqtt_client.hpp"
#include "file_management.hpp"
#include "sleep.hpp"

#define DEBUG

#ifdef DEBUG
#define PRINT_DEBUG(...) Serial.println(__VA_ARGS__)
#else
#define PRINT_DEBUG(...)
#endif

#define ssid             "POC"
#define password         "POWERIoT"

#define MQTT_SERVER      const_cast<char*>("192.168.111.69")
#define MQTT_PORT        4444

#define TIME_SLEEP       2000
#define WIFI_TRIES       35
#define BAT_THRESH       11.65
#define SOLAR_THRESH     20
#define MEASUREMENT_FILE "/measurements.txt"

telemetry_t telemetry   = {0, 0, 0, 0, 0, 0, 0, 0};
char        timeStr[50] = "";
struct tm   currentTime;

/**
 * @brief Intenta realizar la conexion WiFi hasta WIFI_TRIES intentos
 */
static void setup_wifi();

/**
 * @brief Conmuta los reles en funcion de las medidas tomadas
 * 
 * 1. Si la tension del panel solar es mayor que SOLAR_THRESH, se activa el panel solar.
 * 2. Si la tension de la bateria de backup es mayor que BAT_THRESH, se activa la bateria de backup.
 * 3. En caso contrario, se activa la proteccion de sobredescarga.
 * 
 * @param telemetry Puntero a la estructura de medidas
 */
static void setRelays(telemetry_t *telemetry);

void setup() {
    pinMode(RELAY_SW_IN,  OUTPUT);
    pinMode(RELAY_SW_OUT, OUTPUT);
    pinMode(SDA,          OUTPUT);
    pinMode(SCL,          OUTPUT);

    Serial.begin(9600);
}

void loop() {
    measureINA226(&telemetry); // Si no estan los INA conectados, bloquea

    PRINT_DEBUG("[INA] VSolar [V]:  " + String(telemetry.VSolar));
    PRINT_DEBUG("[INA] ISolar [mA]: " + String(telemetry.ISolar));
    PRINT_DEBUG("[INA] VBatbu [V]:  " + String(telemetry.VBatbu));
    PRINT_DEBUG("[INA] IBatbu [mA]: " + String(telemetry.IBatbu));
    PRINT_DEBUG("[INA] VBat1 [V]:   " + String(telemetry.VBat1));
    PRINT_DEBUG("[INA] IBat1 [mA]:  " + String(telemetry.IBat1));
    PRINT_DEBUG("[INA] VBat2 [V]:   " + String(telemetry.VBat2));
    PRINT_DEBUG("[INA] IBat2 [mA]:  " + String(telemetry.IBat2));

    setRelays(&telemetry);

    setup_wifi();

    if (WiFi.status() == WL_CONNECTED){
        configTime(3600, 0, "time.nist.gov", "0.pool.ntp.org");
        
        if (!publishTelemetry(MQTT_SERVER, MQTT_PORT, telemetry)){
            PRINT_DEBUG("[POWER] ERROR publicando medidas");
        } else {
            PRINT_DEBUG("[POWER] Medidas publicadas con exito");
        }

        getLocalTime(&currentTime, 1000);
        strftime(timeStr, 50, "[%d/%m/%y - %H:%M:%S]", &currentTime);
        PRINT_DEBUG("[WIFI] Sincronizado con NTP. Hora obtenida: " + String(timeStr));
    } else {
        PRINT_DEBUG("[WIFI] No se ha podido conectar a la red WiFi");
        snprintf(timeStr, 50, "[00/00/00 - 00:00:00]");
    }

    if (write_meas(MEASUREMENT_FILE, telemetry, timeStr)){
        PRINT_DEBUG("[POWER] Fichero escrito con exito");
    } else {
        PRINT_DEBUG("[POWER] El fichero no ha podido ser escrito");
    }

    sleep_low_power(TIME_SLEEP);
}

static void setup_wifi() {
    delay(10);
    Serial.print("[WIFI] Intentando conectar a " ssid);

    WiFi.begin(ssid, password);
    for (int i = 0; i < WIFI_TRIES && WiFi.status() != WL_CONNECTED; i++) {
        delay(500);
        Serial.print(".");
        i++;
    }
    if(WiFi.status() == WL_CONNECTED){
        PRINT_DEBUG("[WIFI] Conectado a la red WiFi "  ssid  " con IP: "); PRINT_DEBUG(WiFi.localIP());
    } else {
        PRINT_DEBUG("[WIFI] No se ha podido conectar a la red WiFi");
    }
}

static void setRelays(telemetry_t *telemetry){
    if (telemetry->VSolar > SOLAR_THRESH) {
        PRINT_DEBUG("[RELAY] Panel solar activo");

        digitalWrite(RELAY_SW_IN,LOW);
        digitalWrite(RELAY_SW_OUT,LOW);  
    } else if (telemetry->VBatbu > BAT_THRESH){
        PRINT_DEBUG("[RELAY] Bateria backup activa");

        digitalWrite(RELAY_SW_IN,HIGH);
        digitalWrite(RELAY_SW_OUT,HIGH);
    } else {
        PRINT_DEBUG("[RELAY] Activando proteccion sobredescarga. Se detendra el sistema");

        digitalWrite(RELAY_SW_IN,LOW);
        digitalWrite(RELAY_SW_OUT,LOW);
    }
}
\end{lstlisting}

\begin{lstlisting}[language=c++,caption={Fichero\texttt{mqtt\_client.cpp}}, captionpos=b]
#include "mqtt_client.hpp"

static void reconnect(PubSubClient& client, char* const mqtt_server, int mqtt_port);

bool publishTelemetry(char* const mqtt_server, int mqtt_port, telemetry_t& telemetry){
    WiFiClient   wifiClient;
    WiFiUDP      ntpUDP;
    PubSubClient client(wifiClient);
    bool         statusPublish=false;
    char         result[256];

    client.setServer(mqtt_server, mqtt_port);

    if (!client.connected()){
        reconnect(client, mqtt_server, mqtt_port);
    }

    snprintf(result, sizeof(result), 
        "{VSolar:%.2f,ISolar:%.2f,VBatbu:%.2f,IBatbu:%.2f,VBat1:%.2f,IBat1:%.2f,VBat2:%.2f,IBat2:%.2f}", 
        telemetry.VSolar, telemetry.ISolar, 
        telemetry.VBatbu, telemetry.IBatbu, 
        telemetry.VBat1,  telemetry.IBat1, 
        telemetry.VBat2,  telemetry.IBat2
    );

    statusPublish = client.publish("v1/devices/me/telemetry", result);
    client.disconnect();
    return statusPublish;
}

void reconnect(PubSubClient& client, char* const mqtt_server, int mqtt_port) {
    for (int i = 0; i < 3 && !client.connected(); i++){

        Serial.print("[MQTT] Iniciando conexion a " + String(mqtt_server) + ":" + String(mqtt_port));

        if (client.connect("Cliente", "Cliente", "Cliente")) {
            Serial.println("[MQTT] Conectado");
        } else {
            Serial.print("[MQTT] Conexion fallida. Estado = " + String(client.state()) + ". Reintentando en 0.5 segundos...");
            delay(500);
        }
    }
}
\end{lstlisting}

\begin{lstlisting}[language=c++,caption={Fichero\texttt{mqtt\_client.hpp}}, captionpos=b]
/**
 * @file mqtt_client.hpp
 * @author David Andrino
 * @author Estela Mora
 * @author Hugo Sanchez
 * @author Fernando Sanz
 * @brief Modulo de cliente MQTT
 * @version 1.1
 * @date 15-12-2024
 */
#ifndef MQTT_CLIENT_HPP
#define MQTT_CLIENT_HPP
#include <WiFiUdp.h>
#include <ESP8266WiFi.h>
#include <PubSubClient.h>
#include <Wire.h>

#include "telemetry.hpp"

/** 
 * @brief Numero de reintentos de conexion con el servidor MQTT.
 */
#define REINTENTOS_MQTT 3

/**
 * @brief Crea el cliente MQTT y establece la conexion con el servidor MQTT.
 * Ademas, define la estructura del mensaje para publicar las medidas
 * tomadas y mostrarlas en ThingsBoard. En caso de que ocurra un error
 * en la conexion con el servidor MQTT, se reintentara cada 5 segundos
 * mediante una espera bloqueante.
 * @param mqtt_server Direccion del servidor MQTT.
 * @param mqtt_port Puerto del servidor MQTT.
 * @param telemetry Referencia a una estructura que contiene las medidas de telemetria.
 * @return true si la publicacion de telemetria fue exitosa, false en caso contrario.
 */
extern bool publishTelemetry(char* mqtt_server, int mqtt_port, telemetry_t& telemetry);

#endif
\end{lstlisting}

\begin{lstlisting}[language=c++,caption={Fichero\texttt{sleep.cpp}}, captionpos=b]
#include "sleep.hpp"

#include <ESP8266WiFi.h>

void sleep_low_power(int time_delay) {
    Serial.println("[SLEEP] Entrando en modo bajo consumo");
    WiFi.mode(WIFI_OFF);
    WiFi.forceSleepBegin();
    system_update_cpu_freq(SYS_CPU_80MHZ);

    delay(time_delay);

    Serial.println("[SLEEP] Saliendo del modo bajo consumo");
    system_update_cpu_freq(SYS_CPU_160MHZ);
    WiFi.forceSleepWake();
    delay(1);
    WiFi.mode(WIFI_STA);

    // No hace falta conectar a la red WiFi, ya que se conectara en el loop
}
\end{lstlisting}

\begin{lstlisting}[language=c++,caption={Fichero\texttt{sleep.hpp}}, captionpos=b]
/**
 * @file sleep.hpp
 * @author David Andrino
 * @author Estela Mora
 * @author Hugo Sanchez
 * @author Fernando Sanz
 * @brief Modulo de bajo consumo
 * @version 1.1
 * @date 15-12-2024
 */
#ifndef SLEEP_H
#define SLEEP_H

/**
 * @brief Pone al ESP8266 en modo de bajo consumo durante el tiempo que se le indique
 * 
 * @param time_delay Tiempo en milisegundos que el ESP8266 estara en modo de bajo consumo
 */
extern void sleep_low_power(int time_delay);

#endif 
\end{lstlisting}

\begin{lstlisting}[language=c++,caption={Fichero\texttt{telemetry.hpp}}, captionpos=b]
/**
 * @file telemetry.hpp
 * @author David Andrino
 * @author Estela Mora
 * @author Hugo Sanchez
 * @author Fernando Sanz
 * @brief Definicion de la estructura de telemetria
 * @version 1.1
 * @date 15-12-2024
 */
#ifndef TELEMETRY_H
#define TELEMETRY_H

/**
 * @brief Estructura con las medidas tomadas por los diferentes sensores
 */
typedef struct {
    float VSolar;  /**< Tension panel solar */
    float ISolar;  /**< Corriente panel solar */
    float VBatbu;  /**< Tension bateria backup */
    float IBatbu;  /**< Corriente bateria backup */
    float VBat1;   /**< Tension bateria 1 */
    float IBat1;   /**< Corriente bateria 1 */
    float VBat2;   /**< Tension bateria 2 */
    float IBat2;   /**< Corriente bateria 2 */
} telemetry_t;

#endif
\end{lstlisting}
